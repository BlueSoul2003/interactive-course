\documentclass[11pt, a4paper]{article}

% --- PACKAGES ---
\usepackage[a4paper, margin=2cm]{geometry}
\usepackage{amsmath}
\usepackage{amssymb}
\usepackage{enumitem}
\usepackage{fontspec} % Requires XeLaTeX or LuaLaTeX compiler on Overleaf
\usepackage[english]{babel}
\usepackage{xcolor} 
\usepackage{multicol}

% --- FONT SETTINGS ---
\setmainfont{Noto Sans}

% --- CUSTOM COMMANDS ---
\newcommand{\blankline}[1]{\underline{\hspace{#1}}}

\begin{document}

\pagestyle{empty}

% --- NEW SECTION: QUESTIONS 1-10 ---
\noindent\textbf{\large\color[rgb]{0, 0.5, 0.5} Round the following numbers to the nearest ten.}
\begin{multicols}{2}
\begin{enumerate}[label=\arabic*., start=1, leftmargin=1cm]
    \item $771 \approx$ \blankline{2.5cm}
    \item $848 \approx$ \blankline{2.5cm}
    \item $661 \approx$ \blankline{2.5cm}
    \item $296 \approx$ \blankline{2.5cm}
    \item $1,087 \approx$ \blankline{2.5cm}
    \item $1,782 \approx$ \blankline{2.5cm}
    \item $39,917 \approx$ \blankline{2.5cm}
    \item $46,547 \approx$ \blankline{2.5cm}
    \item $11,201 \approx$ \blankline{2.5cm}
    \item $59,999 \approx$ \blankline{2.5cm}
\end{enumerate}
\end{multicols}

\vspace{0.5cm}

% --- NEW SECTION: QUESTIONS 11-20 ---
\noindent\textbf{\large\color[rgb]{0, 0.5, 0.5} Round the following numbers to the nearest hundred.}
\begin{multicols}{2}
\begin{enumerate}[label=\arabic*., start=11, leftmargin=1cm]
    \item $536 \approx$ \blankline{2.5cm}
    \item $881 \approx$ \blankline{2.5cm}
    \item $3,084 \approx$ \blankline{2.5cm}
    \item $1,117 \approx$ \blankline{2.5cm}
    \item $6,944 \approx$ \blankline{2.5cm}
    \item $89,544 \approx$ \blankline{2.5cm}
    \item $23,891 \approx$ \blankline{2.5cm}
    \item $12,057 \approx$ \blankline{2.5cm}
    \item $61,272 \approx$ \blankline{2.5cm}
    \item $74,808 \approx$ \blankline{2.5cm}
\end{enumerate}
\end{multicols}

\vspace{0.5cm}

% --- SECTION 1 (Previously 21-26) ---
\noindent\textbf{\large\color[rgb]{0, 0.5, 0.5} Round the following numbers to the nearest ten and estimate their values.}
\begin{enumerate}[label=\arabic*., start=21, leftmargin=1cm]
    \item $36 + 12 \approx$ \blankline{4cm}
    \item $672 + 48 \approx$ \blankline{4cm}
    \item $66 + 725 \approx$ \blankline{4cm}
    \item $932 - 19 \approx$ \blankline{4cm}
    \item $419 - 38 \approx$ \blankline{4cm}
    \item $519 - 21 \approx$ \blankline{4cm}
\end{enumerate}

\vspace{1cm}

% --- SECTION 2 (Previously 27-34) ---
\noindent\textbf{\large\color[rgb]{0, 0.5, 0.5} Estimate the value.}
\begin{enumerate}[label=\arabic*., start=27, leftmargin=1cm]
    \item $48 \times 8 \approx$ \blankline{4cm}
    \item $25 \times 7 \approx$ \blankline{4cm}
    \item $301 - 9 \approx$ \blankline{4cm}
    \item $697 - 88 \approx$ \blankline{4cm}
    \item $118 \div 4 \approx$ \blankline{4cm}
    \item $324 \div 5 \approx$ \blankline{4cm}
    \item $463 + 93 + 551 \approx$ \blankline{4cm}
    \item $876 + 121 + 43 \approx$ \blankline{4cm}
\end{enumerate}

\vspace{1cm}

% --- SECTION 3 (Previously 35-36) ---
\noindent\textbf{\large\color[rgb]{0, 0.5, 0.5} Fill in each blank with the correct answer.}
\begin{enumerate}[label=\arabic*., start=35, leftmargin=1cm]
    \item $12 = \blankline{1.5cm} \times \blankline{1.5cm}$ \\
          $12 = \blankline{1.5cm} \times \blankline{1.5cm}$ \\
          $12 = \blankline{1.5cm} \times \blankline{1.5cm}$ \\[5pt]
          The factors of 12 are \blankline{1cm}, \blankline{1cm}, \blankline{1cm}, \blankline{1cm}, \blankline{1cm}, and \blankline{1cm}.

    \item $42 = \blankline{1.5cm} \times \blankline{1.5cm}$ \\
          $42 = \blankline{1.5cm} \times \blankline{1.5cm}$ \\
          $42 = \blankline{1.5cm} \times \blankline{1.5cm}$ \\
          $42 = \blankline{1.5cm} \times \blankline{1.5cm}$ \\[5pt]
          The factors of 42 are \blankline{1cm}, \blankline{1cm}, \blankline{1cm}, \blankline{1cm}, \blankline{1cm}, \blankline{1cm}, \blankline{1cm}, and \blankline{1cm}.
\end{enumerate}

\vfill
\noindent {\small Singapore Math Level 4A \& 4B \hfill 27}
\newpage

\begin{enumerate}[label=\arabic*., start=37, leftmargin=1cm]
    \item $36 = \blankline{1.5cm} \times \blankline{1.5cm}$ \\
          $36 = \blankline{1.5cm} \times \blankline{1.5cm}$ \\
          $36 = \blankline{1.5cm} \times \blankline{1.5cm}$ \\
          $36 = \blankline{1.5cm} \times \blankline{1.5cm}$ \\
          $36 = \blankline{1.5cm} \times \blankline{1.5cm}$ \\[5pt]
          The factors of 36 are \blankline{0.8cm}, \blankline{0.8cm}, \blankline{0.8cm}, \blankline{0.8cm}, \blankline{0.8cm}, \blankline{0.8cm}, \blankline{0.8cm}, \blankline{0.8cm}, and \blankline{0.8cm}.

    \item \begin{enumerate}[label=(\alph*)]
        \item The factors of 8 are \blankline{8cm}.
        \item The factors of 16 are \blankline{8cm}.
        \item The common factors of 8 and 16 are \blankline{8cm}.
    \end{enumerate}

    \item \begin{enumerate}[label=(\alph*)]
        \item The factors of 14 are \blankline{8cm}.
        \item The factors of 28 are \blankline{8cm}.
        \item The common factors of 14 and 28 are \blankline{8cm}.
    \end{enumerate}

    \item \begin{enumerate}[label=(\alph*)]
        \item The factors of 9 are \blankline{8cm}.
        \item The factors of 18 are \blankline{8cm}.
        \item The common factors of 9 and 18 are \blankline{8cm}.
    \end{enumerate}

    \item The first four multiples of 5 are \blankline{1cm}, \blankline{1cm}, \blankline{1cm}, and \blankline{1cm}.
    \item The first three multiples of 9 are \blankline{1cm}, \blankline{1cm}, and \blankline{1cm}.
    \item The third multiple of 6 is \blankline{2cm}.
    \item The seventh multiple of 5 is \blankline{2cm}.

    \item \begin{enumerate}[label=(\alph*)]
        \item The first six multiples of 2 are \blankline{1cm}, \blankline{1cm}, \blankline{1cm}, \blankline{1cm}, \blankline{1cm}, and \blankline{1cm}.
        \item The first six multiples of 3 are \blankline{1cm}, \blankline{1cm}, \blankline{1cm}, \blankline{1cm}, \blankline{1cm}, and \blankline{1cm}.
        \item The two common multiples of 2 and 3 are \blankline{2cm} and \blankline{2cm}.
    \end{enumerate}

    \item \begin{enumerate}[label=(\alph*)]
        \item The first six multiples of 4 are \blankline{1cm}, \blankline{1cm}, \blankline{1cm}, \blankline{1cm}, \blankline{1cm}, and \blankline{1cm}.
        \item The first six multiples of 8 are \blankline{1cm}, \blankline{1cm}, \blankline{1cm}, \blankline{1cm}, \blankline{1cm}, and \blankline{1cm}.
        \item The three common multiples of 4 and 8 are \blankline{1cm}, \blankline{1cm}, and \blankline{1cm}.
    \end{enumerate}
\end{enumerate}

\vfill
\noindent {\small 28 \hfill Singapore Math Level 4A \& 4B}
\newpage

% --- SECTION 4 ---
\noindent\textbf{\large\color[rgb]{0, 0.5, 0.5} Write your answers on the lines.}
\begin{enumerate}[label=\arabic*., start=47, leftmargin=1cm]
    \item A train traveled 13,769 km from City A to City B. Then, it traveled another 25,325 km to City C. What was the estimated distance traveled by the train from City A to City C? Round each number to the nearest hundred and estimate the value.
    \begin{flushright}\blankline{4cm}\end{flushright}

    \item A number, when added to 7,982, is 25,000. Round this number to the nearest ten.
    \begin{flushright}\blankline{4cm}\end{flushright}

    \item There were 1,345 books left in a bookstore. If the shopkeeper had sold 7,609 books in the past month, how many books were in the bookstore at first? Round the answer to the nearest ten.
    \begin{flushright}\blankline{4cm}\end{flushright}

    \item In a school, there are 1,124 students in the morning session. There are 259 more students in the afternoon session than in the morning session. How many students are in the school when rounded to the nearest hundred?
    \begin{flushright}\blankline{4cm}\end{flushright}

    \item What is the smallest two-digit number that has only 4 factors?
    \begin{flushright}\blankline{4cm}\end{flushright}

    \item 8 is a factor of number X. It is between 50 and 60. What is number X?
    \begin{flushright}\blankline{4cm}\end{flushright}

    \item Number Y is a multiple of 8. It is between 20 and 30. It is also a factor of 48. What is number Y?
    \begin{flushright}\blankline{4cm}\end{flushright}

    \item If 32 is the fourth multiple of a number, what is the number?
    \begin{flushright}\blankline{4cm}\end{flushright}

    \item The two common multiples of 2 one-digit numbers are 14 and 28. If 1 is not the answer, what are the 2 one-digit numbers?
    \begin{flushright}\blankline{4cm}\end{flushright}
\end{enumerate}

\end{document}
