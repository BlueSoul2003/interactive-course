\documentclass[11pt, a4paper]{article}

% --- PACKAGES ---
\usepackage[a4paper, margin=2cm]{geometry}
\usepackage{amsmath}
\usepackage{amssymb}
\usepackage{enumitem}
\usepackage{fontspec} % Requires XeLaTeX or LuaLaTeX compiler on Overleaf
\usepackage[english]{babel}
\usepackage{xcolor} 
\usepackage{multicol}
\usepackage{tcolorbox} % For the boxed header

% --- FONT SETTINGS ---
\setmainfont{Noto Sans}

% --- CUSTOM COLORS ---
\definecolor{workbookteal}{rgb}{0, 0.65, 0.65}

% --- CUSTOM COMMANDS ---
\newcommand{\blankline}[1]{\underline{\hspace{#1}}}
\newcommand{\answerbox}{\fbox{\phantom{MM}}}

\begin{document}

\pagestyle{empty}

% --- HEADER ---
\begin{center}
    \begin{tcolorbox}[
        colframe=workbookteal, 
        colback=white, 
        width=0.7\textwidth, 
        arc=0mm, 
        boxrule=3pt, 
        halign=center,
        valign=center,
        height=1.5cm
    ]
        \centering \Huge \textbf{\color{workbookteal} REVIEW 1}
    \end{tcolorbox}
\end{center}

\vspace{0.5cm}

% --- SECTION: MULTIPLE CHOICE ---
\noindent\textbf{\large\color{workbookteal} Choose the correct answer. Write its number in the parentheses.}

\begin{enumerate}[label=\arabic*., leftmargin=1cm, itemsep=15pt]
    \item Which of the following shows the correct numeral for seventy-two thousand, eight hundred, forty-five?
    \begin{multicols}{2}
    \begin{enumerate}[label=(\arabic*)]
        \item 72,845
        \item 72,854
        \item 78,245
        \item 78,254 \hfill ( \quad )
    \end{enumerate}
    \end{multicols}

    \item The digit 8 in 28,095 stands for \blankline{1.5cm}.
    \begin{multicols}{2}
    \begin{enumerate}[label=(\arabic*)]
        \item 8 ten thousands
        \item 8 thousands
        \item 8 hundreds
        \item 8 tens \hfill ( \quad )
    \end{enumerate}
    \end{multicols}

    \item Which of the following is \textbf{not} a common multiple of 8 and 6?
    \begin{multicols}{2}
    \begin{enumerate}[label=(\arabic*)]
        \item 18
        \item 24
        \item 72
        \item 144 \hfill ( \quad )
    \end{enumerate}
    \end{multicols}

    \item Round each number to the nearest ten and estimate the value of $1,987 + 5,248$.
    \begin{multicols}{2}
    \begin{enumerate}[label=(\arabic*)]
        \item 7,220
        \item 7,230
        \item 7,240
        \item 7,250 \hfill ( \quad )
    \end{enumerate}
    \end{multicols}

    \item Which of the following is a common factor of 28 and 36?
    \begin{multicols}{2}
    \begin{enumerate}[label=(\arabic*)]
        \item 3
        \item 4
        \item 6
        \item 8 \hfill ( \quad )
    \end{enumerate}
    \end{multicols}

    \item 9,050, \blankline{1.5cm}, 7,030, 6,020, 5,010. What is the missing number in the pattern?
    \begin{multicols}{2}
    \begin{enumerate}[label=(\arabic*)]
        \item 8,020
        \item 8,030
        \item 8,040
        \item 8,050 \hfill ( \quad )
    \end{enumerate}
    \end{multicols}

    \item Which of the following has the greatest value?
    \begin{multicols}{2}
    \begin{enumerate}[label=(\arabic*)]
        \item 2,000 less than 10,000
        \item 2,000 less than 8,000
        \item 2,000 more than 1,000
        \item 2,000 more than 800 \hfill ( \quad )
    \end{enumerate}
    \end{multicols}

    \item Which of the following shows the first four multiples of 7?
    \begin{multicols}{2}
    \begin{enumerate}[label=(\arabic*)]
        \item 7, 14, 20, 27
        \item 7, 14, 21, 28
        \item 7, 14, 28, 35
        \item 7, 14, 28, 42 \hfill ( \quad )
    \end{enumerate}
    \end{multicols}

    \item $49,753 = \blankline{1.5cm} + 9$ thousands $+ 7$ hundreds $+ 5$ tens $+ 3$ ones
    \begin{multicols}{2}
    \begin{enumerate}[label=(\arabic*)]
        \item 400 ten thousands
        \item 40 ten thousands
        \item 4 ten thousands
        \item 4 thousands \hfill ( \quad )
    \end{enumerate}
    \end{multicols}

    \item Estimate the value of $404 \times 9$.
    \begin{multicols}{2}
    \begin{enumerate}[label=(\arabic*)]
        \item 3,600
        \item 3,636
        \item 3,645
        \item 4,000 \hfill ( \quad )
    \end{enumerate}
    \end{multicols}
\end{enumerate}

\vfill
\noindent {\small Singapore Math Level 4A \& 4B \hfill 31}
\newpage

% --- SECTION: FILL IN BLANKS ---
\noindent\textbf{\large\color{workbookteal} Write your answers on the lines.}

\begin{enumerate}[label=\arabic*., start=11, leftmargin=1cm, itemsep=25pt]
    \item Write 49,005 in words.
    \\[10pt] \blankline{\textwidth}

    \item In 94,857, the digit 4 is in the \answerbox{} place. \hfill \blankline{4cm}

    \item Arrange these numbers in ascending order.
    \\[10pt] 15,050, \quad 15,005, \quad 15,500
    \\[10pt] \blankline{\textwidth}

    \item Round each number to the nearest ten and estimate its value.
    \\[10pt] $559 + 19 + 942 = $ \answerbox{} \hfill \blankline{4cm}

    \item The seventh multiple of 9 is \answerbox{}. \hfill \blankline{4cm}

    \item Round 89,091 to the nearest hundred. \hfill \blankline{4cm}

    \item 50 thousands + 90 tens + 7 ones = \answerbox{} \hfill \blankline{4cm}

    \item List all the factors of 45. \hfill \blankline{4cm}

    \item Add 18,360 and 2,598. The digit \answerbox{} is in the thousands place. \hfill \blankline{4cm}

    \item List the first two common multiples of 4 and 6. \hfill \blankline{4cm}
\end{enumerate}

\vfill
\noindent {\small 32 \hfill Singapore Math Level 4A \& 4B}

\end{document}
